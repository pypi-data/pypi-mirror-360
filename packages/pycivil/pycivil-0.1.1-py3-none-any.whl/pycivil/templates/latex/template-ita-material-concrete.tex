%- if glossary
\glsadd{f_{ck}} \glsadd{R_{ck}} \glsadd{f_{ctm}} \glsadd{E_{cm}} \glsadd{f_{cm}}
\glsadd{e_{c2}} \glsadd{e_{cu}} \glsadd{gamma_c} \glsadd{sigma^{max}_{c,car}}
\glsadd{sigma^{max}_{c,qp}} \glsadd{lambda} \glsadd{eta} \glsadd{alpha_{cc}}
\glsadd{alpha_{cc,f}}
%- endif
%- if byCode
Avendo scelto la seguente classe di calcestruzzo e l'ambiente di esposizione:
\begin{align*}
\text{Classe} &: \text{ \fbox{\VAR{classe}}} & \text{Norma} &: \text{ \underline{\VAR{norma}} } & \text{Ambiente} &: \text{ \underline{\VAR{environnment}} }
\end{align*}
si assumono i seguenti valori carattetistici per il materiale
%- else
Per il materiale si scelgono i seguenti valori caratteristici:
%- endif
\begin{align*}
           f_{ck} &= \VAR{fck}          &           f_{cm} &= \VAR{fcm}          &     f_{ctm} &= \VAR{fctm}    &        E_{cm} &= \VAR{Ecm} & \gamma_c &= \VAR{gammac} \\
\sigma^{car}_{cd} &= \VAR{sigmac_max_c} & \sigma^{qp}_{cd} &= \VAR{sigmac_max_q} &      e_{cu} &= \VAR{ecu}     &        e_{c2} &= \VAR{ec2} \\
          \lambda &= \VAR{lambda}       &             \eta &= \VAR{eta}          & \alpha_{cc} &= \VAR{alphacc} & \alpha^f_{cc} &= \VAR{alphacc_fire} &
\end{align*}
