%- if shape == 'RECT'
La sezione ha forma \textit{rettangolare} con le seguenti caratteristiche geometriche sono:

\begin{align*}
b &= \VAR{ width } \text{ \textit{mm}} & h &= \VAR{ height } \text{ \textit{mm}}
\end{align*}

ed è rappesentata nella seguente figura:

\begin{figure}[!h]
    \centering

    \def\s{\VAR{ scale }}
    \def\dimdist{ \VAR{ dimDist }}
    \def\rectW{ \VAR{ width } }
    \def\rectH{ \VAR{ height } }

    \begin{tikzpicture}[scale=\s*10, x=1mm, y=1mm]
    %\begin{tikzpicture}[scale=\s, show background rectangle]
    %%%%%%%%%%%%%%%%%%%%%%%%%%%%
    %    CONCRETE SECTION
    %%%%%%%%%%%%%%%%%%%%%%%%%%%%
    \tikzstyle{dimstyle} = [{<[scale=3.0, length=3, width=4]}-{>[scale=3.0, length=3, width=4]}, color=black!100, fill=black!5, thin]
    \tikzstyle{steelstyle} = [color=black!100, fill=black!20, thick]


    \coordinate (TL) at ( \VAR{xTL} , \VAR{yTL} );
    \coordinate (TR) at ( \VAR{xTR} , \VAR{yTR} );
    \coordinate (BL) at ( \VAR{xBL} , \VAR{yBL} );
    \coordinate (BR) at ( \VAR{xBR} , \VAR{yBR} );

    \draw[color=black!100, fill=black!5, thick] (TL) -- (TR) -- (BR) -- (BL) -- (TL);
    \draw[dimstyle] ($(BL) - (0,\dimdist)$) -- ($(BR) - (0,\dimdist)$) node [midway, fill=blue!5] {\rectW};;
    \draw[dimstyle] ($(TR) + (\dimdist,0)$) -- ($(BR) + (\dimdist,0)$) node [midway, fill=blue!5] {\rectH};;

    %%%%%%%%%%%%%%%%%%%%%%%%%%%%
    %       STEEL REBAR
    %%%%%%%%%%%%%%%%%%%%%%%%%%%%
    %- for r in rebars
    \def\tag{ \VAR{r.id} }
    \coordinate  (\tag) at ( \VAR{r.xPos} , \VAR{r.yPos} );
    \def\d{ \VAR{r.diam} }
    \filldraw[steelstyle] (\tag) circle [radius = \d/2]
    node[above=\d/2*\s, anchor=south] {\footnotesize{\tag}};
    %- endfor

    \end{tikzpicture}.
\end{figure}

Essendo la sezione rettangolare appartenente comunque alle sezioni di tipo poligonale
si riportano le coordinate dei vertici:

\begin{longtable}[c]{|c|>{\centering\arraybackslash}p{30mm}|>{\centering\arraybackslash}p{30mm}|}
\caption{Tabella dei vertici \label{tab:vertices}} \\
\hline
\multirow{2}{*}{\textbf{id}} & $\boldsymbol{X}$            & $\boldsymbol{Y}$ \bigstrut \\ \cline{2-3}
                             & \footnotesize{\textit{[mm]}} & \footnotesize{\textit{[mm]}} \\
\endfirsthead

\multicolumn{3}{c}{continuazione Tabella \ref{tab:vertices}}\\
\hline
\multirow{2}{*}{\textbf{id}} & $\boldsymbol{X}$            & $\boldsymbol{Y}$ \bigstrut \\ \cline{2-3}
                             & \footnotesize{\textit{[mm]}} & \footnotesize{\textit{[mm]}} \\
\endhead

\hline
$TL$ & \VAR{xTL} & \VAR{yTL} \\\hline
$TR$ & \VAR{xTR} & \VAR{yTR} \\\hline
$BL$ & \VAR{xBL} & \VAR{yBL} \\\hline
$BR$ & \VAR{xBR} & \VAR{yBR} \\\hline
\end{longtable}
%- endif

Di seguito la tabella relativa al posizionamento degli acciai, essendo $d$ il diametro ed $A$ l'area:

\begin{longtable}[c]{|c|>{\centering\arraybackslash}p{25mm}|>{\centering\arraybackslash}p{25mm}|>{\centering\arraybackslash}p{25mm}|>{\centering\arraybackslash}p{25mm}|}
\caption{Tabella dei rinforzi \label{tab:rebars}} \\
\hline
\multirow{2}{*}{\textbf{id}} & $\boldsymbol{X}$            & $\boldsymbol{Y}$ & $\boldsymbol{d}$ & $\boldsymbol{A}$ \bigstrut \\ \cline{2-5}
                             & \footnotesize{\textit{[mm]}} & \footnotesize{\textit{[mm]}} & \footnotesize{\textit{[mm]}} & \footnotesize{\textit{[mmq]}} \\
\endfirsthead

\multicolumn{5}{c}{continuazione Tabella \ref{tab:rebars}}\\
\hline
\multirow{2}{*}{\textbf{id}} & $\boldsymbol{X}$            & $\boldsymbol{Y}$ & $\boldsymbol{d}$ & $\boldsymbol{A}$ \bigstrut \\ \cline{2-5}
                             & \footnotesize{\textit{[mm]}} & \footnotesize{\textit{[mm]}} & \footnotesize{\textit{[mm]}} & \footnotesize{\textit{[mmq]}} \\
\endhead

%- for r in rebars
\hline
\VAR{r.id} & \VAR{r.xPos} & \VAR{r.yPos} & \VAR{r.diam} & \VAR{r.area} \\
%- endfor

\hline
\end{longtable}
