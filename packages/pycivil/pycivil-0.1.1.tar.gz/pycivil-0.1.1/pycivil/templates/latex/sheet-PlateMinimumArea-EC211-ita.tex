\centering

\begin{figure}[h!]
	\centering
	\begin{tabular}{ll}\toprule
		\multicolumn{2}{c}{\textbf{Dati generali}}\\\midrule
		Elemento strutturale: & \textbf{ \VAR{elementDescr} }  \\
		Norma utilizzata per il materiale: & \textbf{ \VAR{keyCode} }  \\
		\bottomrule
	\end{tabular}
\end{figure}

\begin{figure}[h!]
	\centering
	\begin{tabular}{lll}\toprule
		\multicolumn{3}{c}{\textbf{Caratteristiche dei materiali}}\\ \midrule
		Descrizione & Valore   & \mbox{u.d.m.} \\ \midrule
		Classe del calcestruzzo: & \textbf{ \VAR{concreteClass} } & $\ldots$ \\
		Resistenza cilindrica a compressione: & $f_{ck}=\VAR{cls_fck}$  & $N/mm^2$\\
		Resistenza media a trazione: & $f_{ctm}=\VAR{cls_fctm}$  & $N/mm^2$\\
		Classe di acciaio: &  \textbf{ \VAR{steelClass} }  & $\ldots$\\                        
		Resistenza caratteristica a snervamento: & $f_{yk}=\VAR{steel_fyk}$ & $N/mm^2$\\                                
		\bottomrule
	\end{tabular}
\end{figure}

\begin{figure}[h!]
	\centering
	\begin{tabular}{lll}\toprule
		\multicolumn{3}{c}{\textbf{Dati geometrici della sezione}}\\ \midrule
		Descrizione & Valore   & \mbox{u.d.m.} \\ \midrule
		Altezza dell'elemento: &  $H=\VAR{hEl}$  & $mm$ \\
		Diametro delle barre principali: & $\phi_{lt} = \VAR{rebarD}$  &  $mm$ \\
		Diametro delle barre secondarie: &  $\phi_{lt}^{*} = \VAR{rebarDSec}$  & $mm$ \\
		Copriferro delle barre principali: &  $c = \VAR{cover}$  & $mm$ \\
		Copriferro delle barre secondarie: &  $c^*=\VAR{coverSec}$  & $mm$\\ \midrule
		Diametro barre trasversali: &  $\phi_{st}=\VAR{stirrupD}$  & $mm$\\
		Numero di bracci trasversali: & $n_{b,t}=\VAR{nbLegDirX}$  & $\ldots$\\				
		\bottomrule
	\end{tabular}
\end{figure}

\begin{figure}[h!]
\centering
\begin{tabular}{p{4cm}ll} \toprule
	\multicolumn{3}{c}{\textbf{Armatura a flessione minima in direzione principale}} \\ 
    \midrule
	Descrizione & Valore & \mbox{u.d.m.} \\	
    \midrule
    %%%%%%%%%%%%%%%%%%%%%%%%%%%%%%%%%%%%%%%%%%%%%%%%%%%%%%%%%%%%%%%%%%%%
	Altezza utile & $d=H-c-\cfrac{\phi_{lt}}{2}=\VAR{heightUtil}$  & $mm$\\ 
	Area utile & $A_u=b_{t,med} \cdot d=1000 \cdot d = \VAR{areaUtil}$  & $mm^2$\\
	Area minima criterio (1) & $A_{min}^{(1)}=0.26 \cdot \cfrac {f_{ctm}} {f_{yk}}\cdot A_u = \VAR{minimumRebarAreaCrit1}$ \marginnote{\S 9.3.1.1 (1)} & $mm^2$\\
	Area minima criterio (2) & $A_{min}^{(2)}=0.0013 \cdot A_u= \VAR{minimumRebarAreaCrit2}$ \marginnote{\S 9.3.1.1 (1)} & $mm^2$\\
  	Area minima & $A_{min}=max(A_{min}^{(1)},A_{min}^{(2)})=\VAR{minimumRebarArea}$ & $mm^2$\\  
  	Distanza massima in zona corrente & $s_{max,slabs}=min(3H, 400)=\VAR{distMaxRebar}$ \marginnote{\S 9.3.1.1 (3)} & $mm$\\  
	Distanza massima con momento massimo & $s_{max,slabs}^m=min(2H, 250)=\VAR{distMaxRebarMaxLoad}$  \marginnote{\S 9.3.1.1 (3)} & $mm$\\ 
	Area di acciaio in zona corrente & $A_{disp}=\VAR{disposedRebarArea}$  & $mm^2$\\
	Numero di barre in zona corrente & $n_{disp}=\VAR{disposedRebarNumber}$  &  $\ldots$\\
  	Area di acciaio con momento massimo & $A_{disp}^m=\VAR{disposedRebarAreaMaxLoad}$ & $mm^2$\\    
  	Numero di barre con momento massimo & $n_{disp}^m=\VAR{disposedRebarNumberMaxLoad}$  &  $\ldots$\\                    
    %%%%%%%%%%%%%%%%%%%%%%%%%%%%%%%%%%%%%%%%%%%%%%%%%%%%%%%%%%%%%%%%%%%%
    \bottomrule
\end{tabular}
\end{figure}

\begin{figure}[h!]
\centering
\begin{tabular}{p{4cm}ll} \toprule
	\multicolumn{3}{c}{\textbf{Armatura a flessione minima in direzione secondaria}} \\ 
    \midrule
	Descrizione & Valore & \mbox{u.d.m.} \\	
    \midrule
    %%%%%%%%%%%%%%%%%%%%%%%%%%%%%%%%%%%%%%%%%%%%%%%%%%%%%%%%%%%%%%%%%%%%
	Altezza utile & $d^*=H^*-c^*-\cfrac{\phi_{lt}^*}{2}=\VAR{heightUtilSec}$  & $mm$\\ 
	Area utile & $A_u^*=b_{t,med}^* \cdot d^*=1000 \cdot d^* = \VAR{areaUtilSec}$  & $mm^2$\\
	Area minima criterio (1) & $A_{min}^{*(1)}=0.26 \cdot \cfrac {f_{ctm}} {f_{yk}}\cdot A_u^* = \VAR{minimumRebarAreaCrit1Sec}$ \marginnote{\S 9.3.1.1 (1)} & $mm^2$\\
	Area minima criterio (2) & $A_{min}^{*(2)}=0.0013 \cdot A_u^*= \VAR{minimumRebarAreaCrit2Sec}$ \marginnote{\S 9.3.1.1 (1)} & $mm^2$\\
  	Area minima & $A_{min}^*=max(A_{min}^{*(1)},A_{min}^{*(2)})=\VAR{minimumRebarAreaSec}$ & $mm^2$\\  
  	Distanza massima in zona corrente & $s_{max,slabs}^*=min(3.5 H^*, 450)=\VAR{distMaxRebarSec}$ \marginnote{\S 9.3.1.1 (3)} & $mm$\\  
	Distanza massima con momento massimo & $s_{max,slabs}^{*m}=min(3 H^*, 400)=\VAR{distMaxRebarMaxLoadSec}$  \marginnote{\S 9.3.1.1 (3)} & $mm$\\ 
	Area di acciaio in zona corrente & $A_{disp}^*=\VAR{disposedRebarAreaSec}$  & $mm^2$\\
	Numero di barre in zona corrente & $n_{disp}^{*}=\VAR{disposedRebarNumberSec}$  &  $\ldots$\\
  	Area di acciaio con momento massimo & $A_{disp}^{*m}=\VAR{disposedRebarAreaMaxLoadSec}$ & $mm^2$\\    
  	Numero di barre con momento massimo & $n_{disp}^{*m}=\VAR{disposedRebarNumberMaxLoadSec}$  &  $\ldots$\\                    
    %%%%%%%%%%%%%%%%%%%%%%%%%%%%%%%%%%%%%%%%%%%%%%%%%%%%%%%%%%%%%%%%%%%%
    \bottomrule
\end{tabular}
\end{figure}

\begin{figure}[h!]
\centering
\begin{tabular}{p{4cm}ll} \toprule
	\multicolumn{3}{c}{\textbf{Armatura a taglio minima (spilli)}} \\ 
    \midrule
	Descrizione & Valore & \mbox{u.d.m.} \\	
    \midrule
    %%%%%%%%%%%%%%%%%%%%%%%%%%%%%%%%%%%%%%%%%%%%%%%%%%%%%%%%%%%%%%%%%%%%
	Area minima per metro di elemento & $\cfrac{A_{sw,min}}{s} = 0.08 \cdot \cfrac{\sqrt{f_{ck}}}{{f_{yk}}}=\VAR{minimumRebarAreaForElementLenght}$ \marginnote{\S 9.3.2 (2)} & $mm^2$\\ 
	Passo massimo trasversale dei bracci & $s_{t,max}=1.5 \cdot d=\VAR{maxStepTrasv}$  \marginnote{\S 9.3.2 (5)} & $mm$\\
	Numero di bracci complessivo per garantire area minima & $n_{{b}}=\VAR{legsNumber}$  & $\ldots$\\
	Numero di bracci trasversali calcolato & $n_{b,c}=\VAR{legsNumberTrasv}$  & $\ldots$\\
  	Passo massimo per disporre area minima (1) & $s_{max}^{(1)}=\cfrac{n_{b,t}}{n_{b}}\cdot 1000=\VAR{maxStepLongCrit1}$ & $mm$\\  
    Passo massimo con altezza utile (2) & $s_{max}^{(2)}=0.75 \cdot d \cdot (1+ \cot \alpha)=\VAR{maxStepLongCrit2}$ \marginnote{\S 9.3.2 (4)} & $mm$\\  
	Passo massimo in generale & $s_{max}=min(s_{max}^{(1)},s_{max}^{(2)})=\VAR{maxStep}$  & $mm$\\ 
	Area staffe disposta per metro di elemento & $\cfrac{A_{sw}}{s} = \VAR{rebarAreaForElementLenght}$  & $mm^2$\\
    %%%%%%%%%%%%%%%%%%%%%%%%%%%%%%%%%%%%%%%%%%%%%%%%%%%%%%%%%%%%%%%%%%%%
    \bottomrule
\end{tabular}
\end{figure}